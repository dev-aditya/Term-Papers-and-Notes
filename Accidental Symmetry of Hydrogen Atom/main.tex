\documentclass{beamer} 
\usepackage{babel}
\usepackage{caption}
\usepackage{xcolor}
\usepackage{amsmath}
\usepackage{amssymb}
\usepackage{epstopdf}
\usepackage{graphicx}
\usepackage{soul}
\usepackage[utf8]{inputenc}

\usepackage{hyperref}
\usepackage{xstring}
\graphicspath{{./images/}}


\PassOptionsToPackage{unicode}{hyperref}
\PassOptionsToPackage{naturalnames}{hyperref}
% \captionsetup[table]{font=scriptsize}

\usepackage{booktabs}
\usepackage{csquotes}
%%
% load layout
\usepackage{theme}
\setUPCLayout{draft,newlogo}
\newcommand{\nologo}{\setbeamertemplate{logo}{}} 

%%%%%%%%%%%%%%%%%%%%%%%%%%%%%%%%%%%%%%%%%%%%%%%%%%%%%%%%%%%%
\usepackage[backend=biber,  style=alphabetic,  sorting=nty,]{biblatex}
\addbibresource{ref.bib}
% Info %%%%%%%%%%%%%%%%%%%%%%%%%%%%%%%%%%%%%%%%%%%%%%
%%%%%%%%%%%%%%%%%%%%%%%%%%%%%%%%%%%%%%%%%%%%%%%%%%%%%%%%%%%%
	% title
		\title{Accidental Symmetry of Hydrogen Atom}	
	% author 
    % (In the mandatory argument "{}", separate multiple
    % authors with "\and" - use "\\" for better author name formatting
    % in the title page. In the optional argument "[]" include all
	% author names, with no "\and" or text formatting macros.)
	% Example: 
    \author[]{Aditya Dev \\ MS19022}
	%\author[Abbr]{Aditya Dev}
    % Address
   \subtitle{\textsc{IDC412 Seminar Delivery}}
   \logo{\AddToShipoutPictureFG{
      \AtPageMyLowerLeft{{\includegraphics[height=0.7cm]{images/iiser_logo.png}}}
    }}
	\institute{{Email: ms19022@iisermohali.ac.in\\}
	\vspace{10pt}
	\textsc{{\includegraphics[height=2cm]{images/iiser_logo.png}} \\\vspace{5pt}
	    Indian Institute of Science Education and Research} \\ 
	    Mohali 
         }
	% date
		\presentationDate{15 February, 2023}
%%%%%%%%%%%%%%%%


\begin{document}

% typeset front slides

\typesetFrontSlides


%%%%%%%%%%%%%%%%%%%%%%%%%%%%%%%%%%%%%%%%%%%%%%%
%
%   SECTION 1
%
%%%%%%%%%%%%%%%%%%%%%%%%%%%%%%%%%%%%%%%%%%%%%%%

\section{Introduction}

\subsection{The Hydrogen Atom!}


\begin{frame}{Degeneracy of Hydrogen and Symmetry of the Hamiltonian}

In quantum mechanics, the bound-state solution of the hydrogen atom is known to have a $n^2$ degeneracy in the energy levels.  The Hamiltonian is given as:
\begin{equation}\label{eq:hamiltonian}
\hat{H} = \dfrac{\Vec{p}^2}{2m} - \dfrac{k \Vec{r}}{r}
\end{equation}
We also know that in quantum mechanics, ``degeneracy implies symmetry". Observe that the Hamiltonian is invariant under rotations in 3D. 
\end{frame}
\begin{frame}{$SO(3)$ and Rotational Invariance}
 For example; invariance under rotation in 3D (associated group $SO(3)$) leads to conservation of angular momentum, which translates into $2 l +1$ degeneracy of angular momentum eigenstates. 

We notice $\sum_{l = 00}^{n-1} (2 l +1) = n$. And the hydrogen atom appears to have degeneracy 
$$n^2 = \sum_{l_1 = 0}^{n-1} \sum_{l_2 = 0}^{n-1}(2 l_1 +1)(2 l_2 +1)$$
 the occurrence of extra $(2l +1)$ hints at a larger symmetry group. 
 \begin{center}
     BUT HOW DO WE FIND THIS SYMMETRY GROUP?
 \end{center}
\end{frame}

\subsection{Motivation from classical mechanics and  LRL vector}
\begin{frame}{Classical Kepler Problem and LRL Vector}
Let us take some help from the old man (the classical mechanics). 

	In classical Kepler's problem (remember, it has the same Hamiltonian!), we come across a conserved vector quantity termed the ``Laplace- Runge-Lenz"  (LRL) vector which is defined as
	\begin{equation}\label{eq:lrl-cm}
	    \Vec{A} = \dfrac{\Vec{p} \times \Vec{L}}{m} - \dfrac{k \Vec{r}}{r}
	\end{equation}
	and it is easy to show that $\frac{d \Vec{A}}{dt} = \{\Vec{A}, H\} = 0$. Hence, it is a conserved vector quantity. 
\end{frame}


\section{The Quantum Mechanical Treatment Of LRL vector}
\subsection{The Pauli approach to Hydrogen atom problem.}
\begin{frame}{Quantum Mechanics LRL Vector}
    In quantum mechanics, we define a Hermitian version of the LRL vector, and we define it as:
    \begin{equation}
        \label{eq:lrl-qm}
        \Vec{A} = \dfrac{1}{2m_2} \left[ \Vec{p} \times \Vec{L} - \Vec{L} \times \Vec{p}  \right] -\dfrac{k \Vec{r}}{r}
    \end{equation}
    It can be shown that the algebra it follows is given as
    \begin{equation*}
    \begin{aligned}
     [L_i, H] = 0 &&  [A_i, H] = 0
    \end{aligned}
    \end{equation*}
    \begin{equation}\label{eq:commut-unmod}
    \begin{gathered}
       [L_i, A_j] = i \hbar \epsilon_{ijk} A_k\\
        [L_i, L_j] = i \hbar \epsilon_{ijk} L_k \\
        [A_i, A_j] = -i \hbar \epsilon_{ijk} L_k \frac{2}{m_2} H
    \end{gathered}
    \end{equation}
    \textit{Same algebra is followed by Lorentz Group generators, but with a +ve sign for last commutator.}
\end{frame}

\begin{frame}{Some more useful identities}
\begin{itemize}
    \item Some useful relations:
    \begin{equation}\label{eq:useful-relations}
    \begin{gathered}
        \Vec{A} \cdot \Vec{L} = \Vec{L} \cdot \Vec{A} = 0\\
        A^2 = \Vec{A} \cdot \Vec{A} = k^2 + \dfrac{2}{m_e} H (L^2  + \hbar^2)
    \end{gathered}
    \end{equation}
    We will need them soon to see some beautiful mathematics!!.
\end{itemize}
\end{frame}
\subsection{The $SO(4)$ algebra.}
\begin{frame}{SO(4) hyperspherical symmetry of bound state solutions}
    Suppose that there exists a bound state with energy $E < 0$ for the Hamiltonian given in Eq.\ref{eq:hamiltonian}. Let $\mathcal{H}(E)$ be the eigenspace in $\mathcal{H}$ with energy value $E$. We will restrict the action of all of our operators to this eigenspace, and we can multiply Eq. \ref{eq:lrl-qm} with a dimensionful constant to make it the same dimensionally as the angular momentum. In this subspace, we define
    \begin{equation}
        \label{eq:lrl-angl}
        \Tilde{A_i} = \sqrt{\frac{-m}{2E}} A_i
    \end{equation}
    we see that $[\Tilde{A_i}] = [\hbar] = [L_i]$. 
\end{frame}

\begin{frame}{}
    Using $\Tilde{A_i}$ we define:
    \begin{equation}
        \begin{gathered}
            T_i = \frac{1}{2} (L_i + \Tilde{A}_i)\\
            S_i = \frac{1}{2} (L_i - \Tilde{A} _i)
        \end{gathered}
    \end{equation}
    and using the commutation relation form Eq. \ref{eq:commut-unmod}, we get
    \begin{equation}
    \begin{gathered}
    [E, T_i] = [E, S_i] = 0\\
    T^2 = S^2  \\
    [T_i, S_j] = 0 \\
    [T_i, T_j] = i \hbar\epsilon_{ijk} T_k  \\ 
    [S_i, S_j] = i \hbar\epsilon_{ijk} S_k
    \end{gathered}
    \end{equation}
\end{frame}
\begin{frame}{}
    From the above commutation relations, it looks like we have two independent ``angular momentum"-like operators (or algebra) in the eigenspace. And if that is the case, then the whole algebra must be: 
    \[SO(3) \oplus  SO(3) \cong SO(4)\]
    and this INDEED IS THE CASE!! 
    
    Hence, the "accidental symmetry" or "dynamical group" of the hydrogen atom problem is $SO(4)$. Or in other words, the solution set of hydrogen atoms is invariant under rotations in some fictitious 4D. $SO(4)$ is the four dimensional rotation group but does not represent a physical rotation for Hydrogen $\longrightarrow$ this is called a dynamical symmetry
\end{frame}
\subsection{Energy Eigenvalues of the Hydrogen Atom}
\begin{frame}{Energy eigenvalues using the $T^2$  operator}
Since the above problem has ``angular momentum"-like operators, we can do our usual Schwinger approach of ladder operators for ``angular momentum". Performing the same exercise as in the case of angular momentum or harmonic oscillator. We get for some $\psi \in \mathcal{H}(E)$, where $\psi$ is the simultaneous eigenstate of $T, S_z, T_z$
    \begin{equation}
        \label{eq:t-sq}
        T^2  \psi = t (t+1) \hbar^2 \psi, \text{  ,where }t \in \{ 0, \frac{1}{2}, 1, \hdots\}
    \end{equation}
    
    Using Eq. \ref{eq:lrl-angl} and Eq. \ref{eq:useful-relations} we can check that
    \begin{equation}
        \Tilde{A}^2 + L^2 = 4T^2 = \hbar^2 - \frac{k^2 m_e}{2E}
    \end{equation}
\end{frame}
\begin{frame}{}
    i.e., the action of $T^2$ on $\mathcal{H}(E)$ is simply multiplication by the $\frac{\hbar^2}{4} - \frac{k^2 m_e}{8E}$ and by our claim from Eq. \ref{eq:t-sq} we need
    \begin{equation}
        \begin{gathered}
            t(t+1) \hbar^2 = \frac{\hbar^2}{4} - \frac{k^2 m_e}{8E}\\
            \boxed{E_{2t +1} = \dfrac{-m_e k^2}{2 \hbar^2 (2t +1)^2}}
        \end{gathered}
    \end{equation}
    we identify the principal quantum number $n$ as $2t+1$. Furthermore, for a given value of $t$, our previous claim tells us that there are exactly $(2t +1)^2 = n^2$  linearly independent states in $\mathcal{H}(E_n)$. \nocite{*}
\end{frame}
\section{References}
\begin{frame}{References}

\printbibliography

\end{frame}

%%
\end{document}
