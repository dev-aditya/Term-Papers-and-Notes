\documentclass{beamer} 
\usepackage{xcolor}
\usepackage{amsmath, amsthm, amssymb, amsfonts}
\usepackage{graphicx}
\newcommand{\ket}[1]{\left|#1\right \rangle}
\newcommand{\bra}[1]{\left\langle#1\right |}
\newcommand{\inner}[2]{\left\langle#1|#2\right\rangle}
\renewcommand{\exp}[1]{\mathsf{e}^{#1}}
\usepackage{hyperref}
\graphicspath{{./images/}}
\usepackage{booktabs}
\usepackage{theme}
\setUPCLayout{draft,newlogo}

\newcommand{\nologo}{\setbeamertemplate{logo}{}} 

%%%%%%%%%%%%%%%%%%%%%%%%%%%%%%%%%%%%%%%%%%%%%%%%%%%%%%%%%%%%
% Info %%%%%%%%%%%%%%%%%%%%%%%%%%%%%%%%%%%%%%%%%%%%%%
%%%%%%%%%%%%%%%%%%%%%%%%%%%%%%%%%%%%%%%%%%%%%%%%%%%%%%%%%%%%
		\title{Analogy between Classical Canonical Transformations and Quantum Unitary Evolution}	
	% author 
    % (In the mandatory argument "{}", separate multiple
    % authors with "\and" - use "\\" for better author name formatting
    % in the title page. In the optional argument "[]" include all
	% author names, with no "\and" or text formatting macros.)
	% Example: 
    %\author[A. Author Albert Einstein]{Anthony Author \and Albert Einstein}
	\author[Abbr]{\large{IDC412: Seminar Delivery} \\
  [5pt] {\Large Aditya Dev} \\
  [5pt] \texttt{MS19022}}
    % Address
   \subtitle{\textsc{Genesis of Path Integral Formulation}}
   \logo{\AddToShipoutPictureFG{
      \AtPageMyLowerLeft{{\includegraphics[height=0.4cm,keepaspectratio]{\smalllogo}}}
    }}
	\institute{%{\includegraphics[height=2cm,keepaspectratio]{\fulllogo}} \\[5pt]
 {Department of Physical Sciences}\\
 [5pt]\textsc{Indian Institute of Science Education and Research Mohali} \\
        [5pt]{Email: \texttt{ms19022@iisermohali.ac.in} }
        }
	% date
		\presentationDate{February 24, 2023}
%%%%%%%%%%%%%%%%


\begin{document}

% typeset front slides

\typesetFrontSlides

%%%%%%%%%%%%%%%%%%%%%%%%%%%%%%%%%%%%%%%%%%%%%%%
%
%   SECTION 1
%
%%%%%%%%%%%%%%%%%%%%%%%%%%%%%%%%%%%%%%%%%%%%%%%
    
\section{Introduction}



\begin{frame}{Introduction}
%\framesubtitle{}
\begin{itemize}
    \item  Quantum mechanics is built on the analogy with Hamiltonian theory of classical mechanics.
    \item  Canonical coordinates and momenta have simple quantum analogues which allow for the Hamiltonian theory to be transferred into quantum mechanics in all its details.
    \item There is an alternative formulation for classical dynamics called the Lagrangian method, which is believed to be more fundamental.
    \item The Lagrangian method allows for all the equations of motion to be expressed as the stationary property of a certain action function, which can be expressed relativistically.
\end{itemize}
\end{frame}
\begin{frame}{What's the problem then?}
 \begin{itemize}
     \item It is not possible to take over the classical Lagrangian equations in a direct way in quantum mechanics as partial derivatives of the Lagrangian with respect to coordinates (\textit{as they are operators in QM}) and velocities have no meaning.
    \item The only differentiation process that can be carried out in quantum mechanics is that of forming commutation relation, which leads to the Hamiltonian Poission brackets. 
    \item Thus, a quantum Lagrangian theory must be formulated indirectly. The ideas of the classical Lagrangian theory, rather than the equations, must be adopted.
 \end{itemize}
\end{frame}

\section{Canonical Transformations in Classical Mechanics}
\subsection{Hamilton-Jacobi Theory}
\begin{frame}{The Hamilton-Jacobi Equation}
     A \textbf{canonical transformation} is a change of canonical coordinates \((q, p, t) \to (Q, P, t)\) that preserves the form of Hamilton's equations. The Action Integral over the Lagrangian  \({\mathcal {L}}_{qp}=\mathbf {p} \cdot {\dot {\mathbf {q} }}-H(\mathbf {q} ,\mathbf {p} ,t)\) and  \(
     {\mathcal {L}}_{{QP}}={\mathbf {P}}\cdot {\dot {{\mathbf {Q}}}}-K({\mathbf {Q}},{\mathbf {P}},t)
     \) respectively, obtained by the Hamiltonian via (``inverse") Legendre transformation, both must be stationary\footnote{Canonical transformation.  \href{https://en.wikipedia.org/wiki/Canonical_transformation}{In Wikipedia.}}, i.e.
\begin{equation}
\label{eq:action_variation}
\delta \int _{t_{1}}^{t_{2}}\left[\mathbf {p} \cdot {\dot {\mathbf {q} }}-H(\mathbf {q} ,\mathbf {p} ,t)\right]dt = \delta \int _{t_{1}}^{t_{2}}\left[\mathbf {P} \cdot {\dot {\mathbf {Q} }}-K(\mathbf {Q} ,\mathbf {P} ,t)\right]dt =0
    \end{equation}
    Which implies:
    \begin{equation}
        \label{eq:canonical_hamiltonian}
        p \dot{q} -  H =  P \dot{Q} - K +  \dfrac{d S}{d t}
    \end{equation}
\end{frame}
\begin{frame}{Generating Function and Hamilton's Principal Function}
    We consider the case when the transformation function \(S\) depends on the old and on the new coordinates:
    \begin{equation}
        \label{eq:principle_func}
        S = S(q_1, q_2, \hdots, Q_1, Q_2 \hdots ).
    \end{equation}
     From Eq~\ref{eq:canonical_hamiltonian} and~\ref{eq:principle_func}, and using Hamiltonian equation of motion, it follows
    \begin{equation}
    \label{eq:class_trans_eq1}
    p_r = \dfrac{\partial S}{\partial q_r}, \quad P_r = -\dfrac{\partial S}{\partial Q_r}
\end{equation}
In Hamilton-Jacobi theory where the canonical transformation is such that \(K\) is identically zero i.e. (\(K = 0\)), it can be shown that\footnote{Goldstein, H., Et al. (2001). Classical mechanics third edition. Eq 10.13}: 
\begin{equation}
    \label{eq:action_eq_generator}
    \boxed{S = \int L dt + \mathrm{const.}}, \quad L \mathrm{= Lagrangian}
\end{equation}

    %The contact transformation is defined by the relation between the differentials 
    %\begin{equation}
     %   \sum_{r = 1}^{N} p_r dq_r - \sum_{r = 1}^{N}P_r dQ_r = dS
    %\end{equation}
   
\end{frame}
\section{``Canonical" transformations in Quantum Mechanics}
\subsection{Motivation}
\begin{frame}{Coordinate transformation in Quantum Mechanics}
\begin{itemize}
    \item In the quantum theory we may take a representation in which the q’s are diagonal, and a second representation in which the Q’s are diagonal.
    \item The contact (canonical) transformation thus defined corresponds in quantum mechanics to a unitary transformation from the representation in which the quantities q are ``diagonal" to the representation in which the quantities Q are ``diagonal".
    \item There will be a transformation function  connecting the two representations. 
\end{itemize}
\end{frame}
\subsection{Transition from Classical to Quantum Mechanics}
\begin{frame}{Some Quantum Algebra!!}

To find the transformation function let's \(\mathcal{\hat{\alpha}}\) be a function of dynamical variables in QM, then we may evaluate:
\begin{equation}
    \label{eq:alpha_trans}
    \bra{q}\mathcal{\hat{A}}\ket{Q} = \underbrace{\int dq' \bra{q} \mathcal{\hat{A}} \ket{q'}\inner{q'}{Q}}_{\text{Eq 6.1}} = \underbrace{\int dQ' \inner{q}{Q'} \bra{Q'}\mathcal{\hat{A}} \ket{Q}}_{\text{Eq 6.2}} 
\end{equation}
\textbf{\textit{We shall now show that this transformation function $\inner{q}{Q}$ is the quantum analogue of $\exp{i S/\hbar}$}}
\end{frame}

\begin{frame}{Quantum Theory (cont...)}
    We know that imposing the commutation relation on position and momentum operator lead to the below equations:
    \begin{equation}
        \label{eq:commutator}
        \left[ \hat{q}, \hat{p}\right] = i \hbar \longleftrightarrow \bra{q}\hat{p}\ket{\psi} = -i\hbar \dfrac{\partial \inner{q}{\psi}}{\partial q}
    \end{equation}
    From the definitions in Eq~\ref{eq:alpha_trans}.1 we obtain:
   \begin{equation}
   \label{eq:qr_evo}
    \bra{q}\hat{q_r}\ket{Q} = q_r \inner{q}{Q}
    \end{equation}
    \begin{equation}
        \label{eq:pr_evo}
        \bra{q}{\hat{p_r}}\ket{Q} = -i \hbar \dfrac{\partial \inner{q}{Q}}{\partial q_r}
    \end{equation}
    similarly  from the definitions in Eq~\ref{eq:alpha_trans}.2 we obtain:
    \begin{equation}
    \label{eq:Qr_evo}
    \bra{q}\hat{Q_r}\ket{Q} = Q_r \inner{q}{Q}
    \end{equation}
    \begin{equation}
    \label{eq:Pr_evo}
        \bra{q}{\hat{P_r}}\ket{Q} = i \hbar \dfrac{\partial \inner{q}{Q}}{\partial Q_r}
    \end{equation}
\end{frame}
\begin{frame}{The Postulate!}
    Now it's turn for \textbf{the Postulate!}. In Eq~\ref{eq:pr_evo} we put:
    \begin{equation}
    \label{eq:postulate}
        \boxed{\inner{q}{Q} = \exp{iU/\hbar}}
    \end{equation}
    where \(U\) is a new function of q's and Q's, we get from Eq~\ref{eq:pr_evo}
    \begin{equation}
        \label{eq:pr_postulate}
        \bra{q}{\hat{p_r}}\ket{Q} =  \dfrac{\partial U}{\partial q_r}\inner{q}{Q}
    \end{equation}
    and similarly from Eq~\ref{eq:Pr_evo}, we obtain:
    \begin{equation}
         \label{eq:Pr_postulate}
        \bra{q}{\hat{P_r}}\ket{Q} =  - \dfrac{\partial U}{\partial Q_r}\inner{q}{Q}
    \end{equation}
\end{frame}
\begin{frame}{Ta Daaa!}
    Comparing Eq~\ref{eq:pr_postulate}  \&~\ref{eq:Pr_postulate} with Eq~\ref{eq:class_trans_eq1}
    we see that:
    \begin{equation}
    \boxed{ p_r = \dfrac{\partial U}{\partial q_r}, \quad P_r = -\dfrac{\partial U}{\partial Q_r}}
    \end{equation}
    which are exactly like the canonical transformation!!!! 

    And as shown in Eq~\ref{eq:action_eq_generator} the \(U\) is nothing but the action; hence we get the familiar transition amplitude as
    \begin{equation}
        \boxed{\inner{q}{Q} \sim \exp{{i \over \hbar}\int L dt}} 
    \end{equation}
\end{frame}

\section{Refrences}
\begin{frame}{References}
\nocite{*}
\bibliographystyle{unsrt}
\bibliography{bibliography.bib}
\end{frame}

%%
\end{document}
